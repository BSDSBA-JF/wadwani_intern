This study evaluates AI exposure of Filipino occupations using the My Career Advisor (MCA) Job List from the Wadhwani Foundation, 
mapped to Standard Occupational Classification (SOC) codes of the O*NET database via Philippine Standard Occupational Classification (PSOC) and International Standard Classification of Occupations (ISCO) crosswalks. 
AI Occupational Exposure (AIOE) and a complementarity index were computed following Felten et al.\ (\citeyear{felten2021}) and Pizzinelli et al.\ (\citeyear{Pizzinelli2023Labor}), with 16.9\% of values imputed. 
Jobs were classified into four types: Augmentable, Automatable, Protected, and Isolated. 
Higher-educated workers hold 46\% of Augmentable jobs versus 8\% for less-educated, while 43\% of less-educated jobs are Isolated. 
Sector-level C-AIOE analysis shows financial services, administrative support, and ICT face the highest AI exposure, whereas construction, agriculture, and manufacturing are lowest. 
These findings highlight heterogeneous AI risks across sectors and educational pathways, offering insights for workforce planning.
